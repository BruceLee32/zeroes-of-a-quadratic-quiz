\documentclass{article}

\usepackage{xcolor}
\pagecolor{white}
\usepackage[utf8]{inputenc}
\usepackage{amsmath}
\usepackage{graphicx}
\usepackage{gensymb}
\usepackage{fancyhdr}
\usepackage{lipsum}
\usepackage[margin=0.5in,includefoot]{geometry}
\usepackage{fancyhdr}
\usepackage[normalem]{ulem}
\usepackage{cancel}
\usepackage{amsfonts}
\usepackage{amssymb}
\usepackage{amsbsy}
\usepackage{amsmath}
\usepackage{mathtools}
\usepackage{setspace}

\begin{document}
\section*{Unit 3B Test: Zeros of a Quadratic (2022)}
Grading:\\
knowledge and understanding: ?\\
application: ?\\
thinking: ?\\
communication: 4 + 2 (form)\\

\vskip 1cm

\noindent{(1) Solve the equation $2(x-1)^2-8=0$ [A][3].}
\vskip 3cm
\noindent{(2) Determine the equations of the parabola in vertex form for the following}\\

a.) If the vertex of a parabola is $(3,-5)$ and the y-intercept is at $y=8$ [T][3].
\vskip 3cm
b.) If the parabola $y=x^2$ is stretched by a factor of 5, moved 6 units to the right, and moved 7 units down [T][2]. 
\vskip 3cm
\noindent{(3) Rewrite the equation $y=2x^2+8x+3$ in vertex form and also sketch it [A][3].}
\newpage
\noindent{(4) Sketch the parabola $y=2(x+3)^2-2$. The sketch must include the x-intercept(s), y-intercept, and vertex of the parabola [A][3].}
\vskip 9.5cm
\noindent{(5) For the parabola $y=3(x-5)^2-6$}\\

a.) Identify the equation for the axis of symmetry, the max/min, and the vertex [K/U][3].
\vskip 3.5cm
b.) Describe the transformation of the parabola [C][4].
\vskip 3.5cm
c.) Determine the x-intercept of the parabola [K/U][2].

\end{document}
